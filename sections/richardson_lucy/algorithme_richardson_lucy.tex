\subsection{Algorithme de Richardson-Lucy}

L'algorithme de Richardson-Lucy procède de manière itérative comme suit :

\begin{algorithm}[H]
\caption{Richardson-Lucy Deconvolution}
\KwIn{Image floue $I$, PSF $P$, nombre d'itérations $n\_it$}
\KwOut{Image restaurée $J$}
Initialiser $J_0 = I$\;
\For{$n = 1$ \KwTo $n\_it$}{
    Convoluer $J_n$ avec $P$ pour obtenir une estimation floutée $I_{\text{estimée}}$\;
    $I_{\text{estimée}} = J_n \ast P$\;
    Calculer le ratio de flou relatif $\text{Ratio} = \frac{I}{I_{\text{estimée}} + \epsilon}$\;
    Convoluer ce ratio avec le miroir de la PSF\;
    $\text{Correction} = \text{Ratio} \ast P_{\text{miroir}}$\;
    Mettre à jour l'estimation $J_{n+1} = J_n \times \text{Correction}$\;
}
\Return $J_{n+1}$\;
\end{algorithm}

\subsubsection{Initialisation}
L'algorithme commence par une estimation initiale de l'image défloutée \( J_0 \), souvent prise comme étant l'image observée \( I \).

\subsubsection{Convolution}
À chaque itération, l'estimation actuelle \( J_n \) est convoluée avec la PSF \( P \) pour obtenir une estimation floutée \( I_{\text{estimée}} \).

\begin{equation}
I_{\text{estimée}} = J_n \ast P
\end{equation}

\subsubsection{Ratio de flou relatif}
Le ratio de flou relatif est calculé pour mesurer combien l'estimation floutée diffère de l'image observée.

\begin{equation}
\text{Ratio} = \frac{I}{I_{\text{estimée}} + \epsilon}
\end{equation}

où \( \epsilon \) est une petite constante pour éviter la division par zéro.

\subsubsection{Correction de l'estimation}
Ce ratio est ensuite convolué avec le miroir de la PSF \( P_{\text{miroir}} \) pour corriger l'estimation.

\begin{equation}
\text{Correction} = \text{Ratio} \ast P_{\text{miroir}}
\end{equation}

\subsubsection{Mise à jour de l'estimation}
L'estimation est mise à jour en multipliant l'estimation actuelle par la correction obtenue.

\begin{equation}
J_{n+1} = J_n \times \text{Correction}
\end{equation}

\subsubsection{Explication de l'utilisation du miroir de la PSF}

La convolution avec le miroir de la PSF est essentielle pour corriger l'estimation de l'image. 
Cette étape simule l'effet inverse de la convolution effectuée par la PSF initiale.
De plus, elle s'assure de la symétrie dans la convolution, orientant correctement les mises à jour de l'image.