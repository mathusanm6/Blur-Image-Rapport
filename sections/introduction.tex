\section{Introduction}
Le floutage d'images est un problème courant en photographie et en imagerie scientifique. 
Il peut être causé par divers facteurs tels que le mouvement de la caméra, la mise au point incorrecte, ou la diffusion de la lumière. 
La déconvolution\footnote{La convolution de deux fonctions \( f \) et \( g \) est définie par :
\[
(f * g)(t) = \int_{-\infty}^{\infty} f(\tau) g(t - \tau) \, d\tau
\]}. est une technique utilisée pour inverser le processus de floutage et restaurer l'image originale. Parmi les nombreuses méthodes de déconvolution, l'algorithme de Richardson-Lucy est particulièrement populaire pour sa capacité à produire des résultats de haute qualité, même en présence de bruit.
