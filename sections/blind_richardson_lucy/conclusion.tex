\subsection{Conclusion sur l'algorithme de Richardson-Lucy Aveugle}

Grâce à l'application de l'algorithme de Richardson-Lucy aveugle, nous avons pu restaurer des images floues avec succès en ayant moins de contraintes sur la connaissance de la fonction de floutage.
Toutefois, il est important de noter que l'algorithme est sensible au bruit et peut produire des artefacts si le bruit est trop important.
Il est beaucoup plus sensible aux conditions initiales que l'algorithme de Richardson-Lucy classique, et une mauvaise initialisation peut entraîner une mauvaise restauration de l'image.
La taille du noyau de convolution est également un paramètre critique qui peut affecter la qualité de la restauration.